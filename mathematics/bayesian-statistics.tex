\documentclass[11pt,a4paper]{report}

\usepackage{amsmath}
\usepackage{amssymb}

\begin{document}

\section*{Bayesian statistics}

Bayesian inference differs from more traditional statistical inference
by preserving uncertainty. The Bayesian world-view interprets
probability as measure of believability in an event, that is, how
confident we are in an event occurring.

Bayesians interpret a probability as measure of belief, or confidence,
of an event occurring. A probability is a summary of an opinion. An
individual who assigns a belief of 0 to an event has no confidence
that the event will occur; conversely, assigning a belief of 1 implies
that the individual is absolutely certain of an event occurring.

This philosophy of treating beliefs as probability is natural to
humans. We employ it constantly as we interact with the world and only
see partial truths, but gather evidence to form beliefs.

Consider an uncertain event, for example whether the Artctic ice cap
will have disappeared by the end of the century. These are not events
that can be repeated numerous times in order to define a notion of
probability (i.e. Frequentist statistics). Nevertheless, we will
generally have some idea, for example, of how quickly we think the
polar ice is melting. If we now obtain fresh evidence, for instance
from a new earth observation, we may revise our opinion on the rate of
ice loss. This can all be achieve through Bayesian interpretation of
probability.

\end{document}
